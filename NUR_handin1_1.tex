\section{Poisson Distribution}

In this section, we look at Question 1
where we want to create a function for the Poisson distribution given by
\[ P_{\lambda}(k)=\frac{\lambda^k e^{-\lambda}}{k!}. \]
The problem with this is that when \( k \) is large, we get an overflow while
computing it.
Hence, we must find another way of doing this.

One method is to convert everything in logarithmic space (logspace).
Taking the natural log of \(P_{\lambda}(k)\), we have
\[ \ln{(P)} = \ln{(\lambda^k)} + \ln{(e^{-\lambda})} - \ln{(k!)} \]
\[ \implies \ln{(P)} = k\ln{(\lambda)} - \lambda - \ln{(k!)}. \]
However, \(k! = \prod_{i=1}^k i\), therefore we can convert it to a sum using
logspace since multiplication becomes addition:
\[ \ln{(k!)} = \ln{\left(\prod_{i=1}^k i\right)} = \sum_{i=1}^k\ln{(i)}. \]
The function \texttt{log\_factorial} below takes the log of each element in the
product of the factorial, which are later on all summed up.
Hence, our final formula (see the function
\texttt{poisson} below) is
\[ \ln{(P)} = k\ln{(\lambda)} - \lambda - \sum_{i=1}^k\ln{(i)}. \]
Of course, we need to output it in the normal way, so we convert it back from
logspace using an exponential.

Our script is given by:
\lstinputlisting{NUR_handin1_1.py}

Our script produces the following results:
\lstinputlisting{NUR_handin1_1.txt}

We can see the function works fine and prevents overflows.
